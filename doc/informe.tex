\documentclass{article}
\usepackage[utf8]{inputenc}
\usepackage{xcolor}
\usepackage{geometry}
\usepackage{graphicx}

\newcommand{\set}[1]{\{#1\}}

\title{
  Introducci\'on al procesamiento digital de im\'agenes \\
  {\bf TP 3: Compresi\'on}
}
\author{
  Pablo Barenbaum \\
  Juli\'an Bayardo
}
\date{}

\newcommand{\TODO}[1]{\textcolor{red}{TODO: #1}}

\begin{document}
\maketitle

\section{Introducción}

Este TP consiste en implementar una variante simplificada del método
de compresión JPEG para imágenes digitales.

El método JPEG proviene de un estándar especificado por el comité homónimo,
que data de fines de la década de 1980.
La variante simplificada que se implementa en este trabajo
dista mucho de la complejidad del estándar oficial.
Lo que se implementa es un prototipo minimal,
destinado a entender y poner a prueba sus mecanismos.

Una característica central del método de compresión JPEG es que es {\em lossy},
es decir, con pérdida de información,
lo que resulta en altas tasas de compresión,
con el costo aparejado de reducir la calidad de la imagen.
El componente central del método es la
transformada discreta del coseno (DCT), que representa la imagen
como una matriz de coeficientes en una base de funciones periódicas
(cosenos).
A continuación, los valores de estos coeficientes se truncan:
esta es la única etapa {\em lossy} del método.
Esta transformación es la que permite obtener altas tasas de compresión,
aplicando posteriormente otros métodos de compresión conocidos
(sin pérdida de información) tales como {\em Huffman coding}
o {\em arithmetic coding}.

\section{Implementación}

\TODO{TODO}

\section{Experimentación}

Se eligieron diez imágenes de distinta naturaleza, todas ellas de
$1000 \times 1000$ píxeles. Se muestran en la Figura~\ref{fig:imagenes_de_prueba_gris}.

Como conjunto de parámetros de referencia, se tomaron:
{\bf tamaño de bloque} $B = 8$,
{\bf factor de cuantización} $Q = 50$,
y {\bf umbral de cuantización} $U = 2000$.
Las imágenes resultantes se muestran en
la Figura~\ref{fig:imagenes_de_prueba_comprimidas_8_50_2000}.
Este conjunto de parámetros se utiliza a modo de control,
utilizado para comparar el comportamiento del algoritmo de compresión
cuando se varía alguno de dichos parámetros.

\begin{figure}
\begin{center}
\begin{tabular}[t]{|ll|ll|ll|}
\hline
img0 & \includegraphics[width=4cm]{../imgs/input/imgs_gray/img00.png} &
img1 & \includegraphics[width=4cm]{../imgs/input/imgs_gray/img01.png} &
img2 & \includegraphics[width=4cm]{../imgs/input/imgs_gray/img02.png} \\
\hline
img3 & \includegraphics[width=4cm]{../imgs/input/imgs_gray/img03.png} &
img4 & \includegraphics[width=4cm]{../imgs/input/imgs_gray/img04.png} &
img5 & \includegraphics[width=4cm]{../imgs/input/imgs_gray/img05.png} \\
\hline
img6 & \includegraphics[width=4cm]{../imgs/input/imgs_gray/img06.png} &
img7 & \includegraphics[width=4cm]{../imgs/input/imgs_gray/img07.png} &
img8 & \includegraphics[width=4cm]{../imgs/input/imgs_gray/img08.png} \\
\hline
img9 & \includegraphics[width=4cm]{../imgs/input/imgs_gray/img09.png} &&&& \\
\hline
\end{tabular}
\end{center}
\caption{Imágenes de prueba en tonos de gris}
\label{fig:imagenes_de_prueba_gris}
\end{figure}

\begin{figure}
\begin{center}
\begin{tabular}[t]{|ll|ll|ll|}
\hline
img0 & \includegraphics[width=4cm]{../imgs/output/gray_8_50_2000/img00.png} &
img1 & \includegraphics[width=4cm]{../imgs/output/gray_8_50_2000/img01.png} &
img2 & \includegraphics[width=4cm]{../imgs/output/gray_8_50_2000/img02.png} \\
\hline
img3 & \includegraphics[width=4cm]{../imgs/output/gray_8_50_2000/img03.png} &
img4 & \includegraphics[width=4cm]{../imgs/output/gray_8_50_2000/img04.png} &
img5 & \includegraphics[width=4cm]{../imgs/output/gray_8_50_2000/img05.png} \\
\hline
img6 & \includegraphics[width=4cm]{../imgs/output/gray_8_50_2000/img06.png} &
img7 & \includegraphics[width=4cm]{../imgs/output/gray_8_50_2000/img07.png} &
img8 & \includegraphics[width=4cm]{../imgs/output/gray_8_50_2000/img08.png} \\
\hline
img9 & \includegraphics[width=4cm]{../imgs/output/gray_8_50_2000/img09.png} &&&& \\
\hline
\end{tabular}
\end{center}
\caption{Imágenes comprimidas con parámetros $B=8, Q=50, U=2000$}
\label{fig:imagenes_de_prueba_8_20_5000}
\end{figure}

\newpage
\subsection{Variación del tamaño de bloque}

Para estudiar la variación del tamaño de bloque, se ejecutó el
algoritmo sobre las diez imágenes de prueba con el
factor de cuantización fijo en $Q = 50$,
el umbral de cuantización fijo en $U = 2000$,
y el tamaño de bloque variando con valores
$B \in \set{1,2,4,8,16,32,64,128,256,512}$.

Para cada valor de $B$, en el siguiente gráfico se muestra un
box plot de la tasa de compresión obtenida:\\
\includegraphics[width=10cm]{../imgs/output/gray_plots/b_rate.png}

Para cada valor de $B$, en el siguiente gráfico se muestra un
box plot del {\em peak signal-to-noise ratio} obtenido:\\
\includegraphics[width=10cm]{../imgs/output/gray_plots/b_psnr.png}

Puede apreciarse que hay menor varianza en la tasa de compresión
obtenida a medida que se agranda el tamaño de bloque,
mientras que hay mayor varianza en la calidad de la imagen obtenida.

La mejor tasa de compresión se alcanza con el tamaño de bloque
$B = 8$. El mejor valor de PSNR en promedio se alcanza también
en $B = 8$.

\TODO{TODO}

\newpage
\subsection{Variación del factor de cuantización}

Para estudiar la variación del factor de cuantización, se ejecutó el
algoritmo sobre las diez imágenes de prueba con el
tamaño de bloque fijo en $B = 8$,
el umbral de cuantización fijo en $U = 2000$,
y el factor de cuantización variando con valores
$Q \in \set{12,25,50,100,200,400,800}$.

Para cada valor de $Q$, en el siguiente gráfico se muestra un
box plot de la tasa de compresión obtenida:\\
\includegraphics[width=10cm]{../imgs/output/gray_plots/q_rate.png}

Para cada valor de $Q$, en el siguiente gráfico se muestra un
box plot del {\em peak signal-to-noise ratio} obtenido:\\
\includegraphics[width=10cm]{../imgs/output/gray_plots/q_psnr.png}

Como es de esperarse, la tasa de compresión aumenta a medida que
se aumenta el factor de cuantización, ya que, a mayor valor de $Q$,
los coeficientes se cuantizan en un dominio más chico.
Por otro lado, aumentar el factor de cuantización produce obviamente
una reducción en la calidad de la imagen ya que a mayor valor de $Q$
el método de compresión es más {\em lossy}.

\TODO{TODO}

\newpage
\subsection{Variación del umbral de cuantización}

Para estudiar la variación del umbral de cuantización, se ejecutó el
algoritmo sobre las diez imágenes de prueba con el
tamaño de bloque fijo en $B = 8$,
el factor de cuantización fijo en $Q = 50$,
y el umbral de cuantización variando con valores
$U \in \set{31, 62, 125, 250, 500, 1000, 2000, 4000, 8000, 16000}$.

Para cada valor de $U$, en el siguiente gráfico se muestra un
box plot de la tasa de compresión obtenida:\\
\includegraphics[width=10cm]{../imgs/output/gray_plots/u_rate.png}

Para cada valor de $U$, en el siguiente gráfico se muestra un
box plot del {\em peak signal-to-noise ratio} obtenido:\\
\includegraphics[width=10cm]{../imgs/output/gray_plots/u_psnr.png}


\section{Conclusiones}

\TODO{TODO}

\end{document}

