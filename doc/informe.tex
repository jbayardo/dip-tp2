\documentclass{article}
\usepackage[utf8]{inputenc}
\usepackage{xcolor}

\title{
  Introducci\'on al procesamiento digital de im\'agenes \\
  {\bf TP 3: Compresi\'on}
}
\author{
  Pablo Barenbaum \\
  Juli\'an Bayardo
}
\date{}

\newcommand{\TODO}[1]{\textcolor{red}{TODO: #1}}

\begin{document}
\maketitle

\section{Introducción}

Este TP consiste en implementar una variante simplificada del método
de compresión JPEG para imágenes digitales.

El método JPEG proviene de un estándar especificado por el comité homónimo,
que data de fines de la década de 1980.
La variante simplificada que se implementa en este trabajo
dista mucho de la complejidad del estándar oficial.
Lo que se implementa es un prototipo minimal,
destinado a entender y poner a prueba sus mecanismos.

Una característica central del método de compresión JPEG es que es {\em lossy},
es decir, con pérdida de información,
lo que resulta en altas tasas de compresión,
con el costo aparejado de reducir la calidad de la imagen.
El componente central del método es la
transformada discreta del coseno (DCT), que representa la imagen
como una matriz de coeficientes en una base de funciones periódicas
(cosenos).
A continuación, los valores de estos coeficientes se truncan:
esta es la única etapa {\em lossy} del método.
Esta transformación es la que permite obtener altas tasas de compresión,
aplicando posteriormente otros métodos de compresión conocidos
(sin pérdida de información) tales como {\em Huffman coding}
o {\em arithmetic coding}.

\section{Implementación}

\TODO{TODO}

\section{Experimentación}

\TODO{TODO}

\section{Conclusiones}

\TODO{TODO}

\end{document}

